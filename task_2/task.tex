\section*{Задание 2}

\subsection*{Логическая организация кэш-памяти; тестовое покрытие}

Для кэш-памяти данных с заданными параметрами:

\begin{enumerate}
	\item Описать структуру адреса;
	\item Построить конечный автомат Мили, описывающий набор кэш-памяти (set); выходной алфавит должен позволять различать адреса (теги) запросов;
	\item Описать входную последовательность минимальной длины, приводящую к вытеснению модифицированных данных (исходное состояние: кэш пуст);
	\item Рассчитать достигнутое тестовое покрытие в метрике состояний и в метрике переходов построенного конечного автомата.
\end{enumerate}

Возможные события:

\begin{itemize}
	\item Операции чтения/записи со стороны вычислительного ядра;
	\item Снуп-инвалидатор со стороны подсистемы памяти: при его получении кэш вычёркивает немодифицированный блок с заданным адресом либо вытесняет модифицированный блок в память.
\end{itemize}

Структура кэш-памяти (вариант ''d''):

\begin{itemize}
	\item Разрядность адреса ($W_a$) -- 48;
	\item Размер кэш-памяти ($S$) -- 2 Мбайта;
	\item Размер блока ($B$) -- 64 байта;
	\item Ассоциативность ($A$) -- 32;
	\item Тип записи -- отложенная (write-back);
	\item Политика заведения -- первый свободный блок набора;
	\item Промах по чтению и записи;
	\item Политика заведения -- LRU.
\end{itemize}
